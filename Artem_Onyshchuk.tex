\documentclass{article}
\usepackage[utf8]{inputenc}
\usepackage[ukrainian]{babel}
\usepackage{listings}
\usepackage{xcolor}
\usepackage{geometry}

\geometry{a4paper, margin=2.5cm}

\lstset{
  basicstyle=\small\ttfamily,
  keywordstyle=\color{blue},
  commentstyle=\color{green!60!black},
  stringstyle=\color{red},
  breaklines=true,
  frame=single,
}

\title{Regex}
\author{}
\date{\today}

\begin{document}

\maketitle

\section{Вступ}
Програма реалізує спрощений варіант регулярних виразів з використанням скінченних автоматів. Реалізація базується на об'єктно-орієнтованому підході з використанням абстрактних класів.

\section{Аналіз структури}
Програма складається з:
\begin{itemize}
  \item Абстрактного класу \texttt{State} з методом \texttt{check\_self}
  \item Конкретних класів станів:
    \begin{itemize}
      \item \texttt{StartState} та \texttt{TerminationState}
      \item \texttt{DotState} - відповідає символу "." (будь-який символ)
      \item \texttt{AsciiState} - відповідає конкретному символу
      \item \texttt{StarState} - відповідає оператору "*"
      \item \texttt{PlusState} - відповідає оператору "+"
    \end{itemize}
  \item Основного класу \texttt{RegexFSM} для обробки регулярних виразів
\end{itemize}

\section{Алгоритм роботи}
\subsection{Парсинг виразу}
При ініціалізації \texttt{RegexFSM} відбувається перетворення рядка regex в послідовність об'єктів-станів:
\begin{itemize}
  \item Звичайні символи → \texttt{AsciiState}
  \item Символ "." → \texttt{DotState}
  \item Оператори "*" та "+" модифікують попередній стан
\end{itemize}

\subsection{Перевірка рядка}
Метод \texttt{check\_string} проходиться по стрічці і співставляє з автоматами:
Для кожного типу стану використовується власна логіка перевірки




\section{Висновок}
Програма демонструє базові принципи роботи з регулярними виразами на рівні скінченних автоматів. Реалізація є простою, але робочою.
\end{document}